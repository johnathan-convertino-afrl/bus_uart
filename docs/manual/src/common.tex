\begin{titlepage}
  \begin{center}

  {\Huge BUS\_UART\_LITE}

  \vspace{25mm}

  \includegraphics[width=0.90\textwidth,height=\textheight,keepaspectratio]{img/AFRL.png}

  \vspace{25mm}

  \today

  \vspace{15mm}

  {\Large Jay Convertino}

  \end{center}
\end{titlepage}

\tableofcontents

\newpage

\section{Usage}

\subsection{Introduction}

\par
BUS UART LITE is a core for interfacing over RS232 UART to a bus of choice. The core will process data to and from the UART.
The data can then be accessed over a BUS, currently AXI lite or Wishbone Standard, and processed as needed. All input and output
over the bus goes into FIFOs that is then tied to the AXIS UART core. The following is information on how to use the device
in an FPGA, software, and in simulation.

\subsection{Dependencies}

\par
The following are the dependencies of the cores.

\begin{itemize}
  \item fusesoc 2.X
  \item iverilog (simulation)
  \item cocotb (simulation)
\end{itemize}

\subsubsection{axi\_lite\_uart\_lite Depenecies}
\begin{itemize}
\item dep
	\begin{itemize}
	\item AFRL:utility:helper:1.0.0
	\item AFRL:device:up\_uart:1.0.0
	\item AD:common:up\_axi:1.0.0
	\end{itemize}
\item dep\_tb
	\begin{itemize}
	\item AFRL:simulation:axis\_stimulator
	\item AFRL:utility:sim\_helper
	\end{itemize}
\end{itemize}


\subsubsection{wishbone\_standard\_uart\_lite Depenecies}
\begin{itemize}
\item dep
	\begin{itemize}
	\item AFRL:utility:helper:1.0.0
	\item AFRL:device:up\_uart\_lite:1.0.0
	\item AFRL:bus:up\_wishbone\_standard:1.0.0
	\end{itemize}
\end{itemize}


\subsubsection{up\_uart\_lite Depenecies}
\begin{itemize}
\item dep
	\begin{itemize}
	\item AFRL:utility:helper:1.0.0
	\item AFRL:device\_converter:fast\_axis\_uart:1.0.0
	\item AFRL:buffer:fifo
	\end{itemize}
\end{itemize}


\subsection{In a Project}
\par
First, pick a core that matches the target bus in question. Then connect the BUS UART core to that bus. Once this is complete the UART pins will need
to be routed so they match the UART device or other.

\section{Architecture}
\par
This core is made up of other cores that are documented in detail in there source. The cores this is made up of are the,
\begin{itemize}
  \item \textbf{fast\_axis\_uart} Interface with UART and present the data over AXIS interface (see core for documentation).
  \item \textbf{fifo} Used for RX and TX FIFO instances. Set to 16 words buffer max (see core for documentation).
  \item \textbf{up\_axi} An AXI Lite to uP converter core (see core for documentation).
  \item \textbf{up\_wishbone\_standard} A wishbone standard to uP converter core (see core for documentation).
  \item \textbf{up\_uart\_lite} Takes uP bus and coverts it to interface with the RX/TX FIFOs and the AXIS UART (see module documentation for information \ref{Module Documentation}).
\end{itemize}

For register documentation please see up\_uart in \ref{Module Documentation}

\section{Building}

\par
The BUS UART LITE is written in Verilog 2001. It should synthesize in any modern FPGA software. The core comes as a fusesoc packaged core and can be included in any other core. Be sure to make sure you have meet the dependencies listed in the previous section. Linting is performed by verible using the lint target.

\subsection{fusesoc}
\par
Fusesoc is a system for building FPGA software without relying on the internal project management of the tool. Avoiding vendor lock in to Vivado or Quartus.
These cores, when included in a project, can be easily integrated and targets created based upon the end developer needs. The core by itself is not a part of
a system and should be integrated into a fusesoc based system. Simulations are setup to use fusesoc and are a part of its targets.

\subsection{Source Files}

\subsubsection{axi\_lite\_uart\_lite File List}
\begin{itemize}
\item src
	\begin{itemize}
	\item src/axi\_lite\_uart\_lite.v
	\end{itemize}
\item tb\_cocotb
	\begin{itemize}
	\item {'tb/tb\_cocotb\_axi\_lite.py': {'file\_type': 'user', 'copyto': '.'}}
	\item {'tb/tb\_cocotb\_axi\_lite.v': {'file\_type': 'verilogSource'}}
	\end{itemize}
\end{itemize}


\subsubsection{wishbone\_standard\_uart\_lite File List}
\begin{itemize}
\item src
	\begin{itemize}
	\item src/wishbone\_standard\_uart\_lite.v
	\end{itemize}
\item tb\_cocotb
	\begin{itemize}
	\item {'tb/tb\_cocotb\_wishbone\_standard.py': {'file\_type': 'user', 'copyto': '.'}}
	\item {'tb/tb\_cocotb\_wishbone\_standard.v': {'file\_type': 'verilogSource'}}
	\end{itemize}
\end{itemize}


\input{src/fusesoc/files_up_uart_lite.tex}

\subsection{Targets}

\input{src/fusesoc/targets_axi_lite_uart_lite.tex}

\input{src/fusesoc/targets_wishbone_standard_uart_lite.tex}

\input{src/fusesoc/targets_up_uart_lite.tex}

\subsection{Directory Guide}

\par
Below highlights important folders from the root of the directory.

\begin{enumerate}
  \item \textbf{docs} Contains all documentation related to this project.
    \begin{itemize}
      \item \textbf{manual} Contains user manual and github page that are generated from the latex sources.
    \end{itemize}
  \item \textbf{src} Contains source files for the core
  \item \textbf{tb} Contains test bench files for iverilog and cocotb
    \begin{itemize}
      \item \textbf{cocotb} testbench files
    \end{itemize}
\end{enumerate}

\newpage

\section{Simulation}
\par
There are a few different simulations that can be run for this core.

\subsection{cocotb}
\par
Cocotb is the only method for simulating the various interations of the bus\_UART core. At the moment there is a
axi\_lite, wishbone\_standard, and uP based versions. This is currently set to use icarus as the sim tool for cocotb.

\par
To run the wishbone sim use the command below.
\begin{lstlisting}[language=bash]
fusesoc run --target sim_cocotb AFRL:device:wishbone_standard_uart_lite:1.0.0
\end{lstlisting}

\par
To run the axi\_lite sim use the command below.
\begin{lstlisting}[language=bash]
fusesoc run --target sim_cocotb AFRL:device:axi_lite_uart_lite:1.0.0
\end{lstlisting}

\par
To run the uP sim use the command below.
\begin{lstlisting}[language=bash]
fusesoc run --target sim_cocotb AFRL:device:up_uart_lite:1.0.0
\end{lstlisting}

\newpage

\section{Module Documentation} \label{Module Documentation}

\par
up\_uart\_lite is the module that integrates the AXIS UART core.
This includes FIFO's that have there inputs/outputs for data tied to registers mapped in the uP bus.
The uP bus is the microprocessor bus based on Analog Devices design. It resembles a APB bus in design,
and is the bridge to other buses BUS UART LITE can use. This makes changing for AXI Lite, to Wishbone to whatever
quick and painless.

\par
axi\_lite\_uart\_lite module adds a AXI Lite to uP (microprocessor) bus converter. The converter is
from Analog Devices.

\par
wishbone\_standard\_uart\_lite module adds a Wishbone Standard to uP (microprocessor) bus converter. This
converter was designed for Wishbone Standard only, NOT pipelined.

\vspace{15mm}
\par
The next sections document these modules. up\_uart\_lite contains the register map explained, and what the various bits do.

